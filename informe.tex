\documentclass{article}
\usepackage[utf8]{inputenc}
\usepackage{amsmath,amsfonts,amssymb} % Paquetes para matemáticas
\usepackage{ulem} % Para subrayar texto
\usepackage{geometry}
\usepackage{indentfirst}
\usepackage{graphicx}
\usepackage{float}

\geometry{
    a4paper,
    left=2cm,
    right=2cm,
    top=2.5cm,
    bottom=2.5cm,
    includefoot,
    headheight=15pt,
    headsep=0.5cm,
    footskip=1cm
}

\title{Métodos Computacionales - Trabajo Práctico 1}
\author{Josefina Jahde, Dafydd Jenkins}
\date{\today}

\begin{document}

\maketitle

\section*{Introducción}
El presente informe tiene como objetivo documentar y analizar los desarrollos realizados en el marco del Trabajo Práctico 1 de la materia Métodos Computacionales. A lo largo de los ejercicios planteados, se trabajó con curvas de Bézier de diferentes grados, explorando sus propiedades geométricas y matemáticas, así como su representación computacional. Para esto, se implementaron algoritmos en Python que permiten construir estas curvas y sus correspondientes transformaciones lineales. Adicionalmente, se estudió la continuidad de las curvas compuestas y se verificaron las condiciones necesarias para lograr continuidad C1C1 y C2C2 en las mismas. Este informe detalla paso a paso la resolución de los ejercicios, las fórmulas obtenidas, los gráficos generados, y las conclusiones derivadas de los experimentos realizados.


\section*{Ejercicio 1}
\subsection*{Primer ítem}
La función solicitada se encuentra en el archivo .ipynb adjunto, definida como \textit{f0(t, puntos)}.
\subsection*{Segundo ítem}
Partimos de la formula:

$$
g(t) = (1 - t)f_0(t) + t f_1(t)
$$

La desarrollamos como se muestra en la imagen:

\begin{figure}[H]
    \centering
    \includegraphics[width=0.6\textwidth]{imagenes/1b.jpg}
    \caption{Desarrollo de $g(t)$}
    \label{fig:ejemplo}
\end{figure}

Llegando a la fórmula final:

$$
g(t) = t^2 (p_0 - 2 p_1 + p_2) + t (-2 p_0 + 2 p_1) + p_0
$$

Como se puede ver, la fórmula obtenida es cuadrática con respecto a $t$, por lo tanto, $g(t)$ es una curva de Bézier cuadrática.
\subsection*{Tercer ítem}
El gráfico se generó con el código que se encuentra en la sección "Ejercicio 1". Como puntos de ejemplo se usaron $p_0 = (1, 2)$, $p_1 = (5, 6)$, $p_2 = (2, 10)$. El gráfico resultante se muestra  a continuación:

\begin{figure}[H]
    \centering
    \includegraphics[width=0.6\textwidth]{imagenes/graf_ej1.png}
    \caption{$f_0(t)$, $f_1(t)$ y $g(t)$ con puntos de ejemplo}
    \label{fig:ejemplo}
\end{figure}

Luego de construir $g(t)$ a partir de $f_0(t)$ y $f_1(t)$ podemos decir que estas dos últimas "se combinan" para generar  $g(t)$. A los resultados de $f_0(t)$ y $f_1(t)$ se les aplica nuevamente la misma función (porque $f_0(t)$ y $f_1(t)$ son la misma función aplicada sobre distintos puntos).

Decidimos definir "la función" como $f(t, p_i, p_j) = (1-t) p_i + t p_j$.

Entonces, $f_0(t) = f(t, p_0, p_1)$, $f_1(t) = f(t, p_1, p_2)$ y $g(t) = f(t, f_0(t), f_1(t))$.

Intuimos que para generar una curva de Bézier cúbica, se agregará un cuarto punto $p_3$. El mismo se combinará con $p_2$ para formar una $f_2(t) = f(t, p_2, p_3)$. El resultado de esa función se combinará con el de $f_1(t)$ para formar una función similar a $g(t)$. El resultado de estas dos "$g(t)$" serán utilizados como parámetros nuevamente de $f(t, p_i, p_j)$ para generar la curva de Bézier cúbica.

\section*{Ejercicio 2}
\subsection*{Primer ítem}
Partimos de la definición de $h(t) = (1 - t)g_1(t) + t g_2(t)$. La desarrollamos como se muestra a continuación:

\begin{figure}[H]
    \centering
    \includegraphics[width=0.6\textwidth]{imagenes/2a.png}
    \caption{Desarrollo de $h(t)$}
    \label{fig:ejemplo}
\end{figure}

Llegando a la fórmula final:

$$
h(t) = t^3p_3 + (3t^2-3t^3)p_2 + (3t^3-6t^2+3t)p1 + (3t^2-t^3-3t+1)p0
$$

\subsection*{Segundo ítem}
Para el segundo ítem, utilizamos el código que se puede encontrar en la sección "Ejercicio 2" del archivo .ipynb. El gráfico resultante es el siguiente:

\begin{figure}[H]
    \centering
    \includegraphics[width=0.6\textwidth]{imagenes/graf_2a.png}
    \caption{Coeficientes de $h(t)$}
    \label{fig:ejemplo}
\end{figure}

Para la suma de los coeficientes, vimos que con $t=0.3, t=0.5 y t=0.8$ dicha suma siempre daba 1. Por lo tanto decidimos ver que pasaba con un t genérico. La suma de los coeficientes será:

$$
t^3 + 3t^2-3t^3 + 3t^3-6t^2+3t +3t^2-t^3-3t+1
$$

En dicha suma, los términos dependientes de t se cancelan para toda t, por lo que la suma siempre vale 1.

\subsection*{Tercer ítem}
Se obtuvo, a partir del código en el archivo .ipynb en la sección "Ejercicio 2", el siguiente gráfico:

\begin{figure}[H]
    \centering
    \includegraphics[width=0.6\textwidth]{imagenes/graf_2c.png}
    \caption{Ejemplos de $h(t)$ con puntos de control aleatorios}
    \label{fig:ejemplo}
\end{figure}

\section*{Ejercicio 3}

\subsection*{Primer ítem:}
Para $S_1 = \{v_1\}$, las posibles combinaciones lineales son $c_1 v_1$. Como la suma de los coeficientes debe ser 1, y solo hay un coeficiente, debe valer $c_1 = 1$. Por lo tanto, $conv(S_1) = \{v_1\}$.
 
Para $S_2 = \{v_1, v_2\}$, las posibles combinaciones lineales son $c_1 v_1 + c_2 v_2$. Como la suma de los coeficientes debe ser 1, debe valer $c_1 + c_2= 1 \iff c_1 = 1 - c_2$. Por lo tanto, $conv(S_2) = \{c_1 v_1+ c_2 v_2 \in \mathbb{R}^2 \mid c_1, c_2 \geq 0$ y $ c_1 = 1- c_2\} = \{$[$v_1, v_2$][$1-c_2, c_2$]$^T \in \mathbb{R}^2 \mid c_1, c_2 \geq 0$\}.

Para $S_3 = \{v_1, v_2, v_3\}$, las posibles combinaciones lineales son $c_1 v_1 + c_2 v_2 + c_3 v_3$. Como la suma de los coeficientes debe ser 1, debe valer $c_1 + c_2 + c_3 = 1 \iff c_1 = 1 - c_2 - c_3$ (hay dos variables libres). Por lo tanto, $conv(S_3) = \{c_1 v_1+ c_2 v_2 + c_3 v_3\in \mathbb{R}^2 \mid c_1, c_2, c_3 \geq 0$ y $ c_1 = 1- c_2-c_3\} = \{$[$v_1, v_2, v_3$][$1-c_2 - c_3, c_2, c_3$]$^T\in \mathbb{R}^2 \mid c_1, c_2, c_3 \geq 0$\}.

Visualizaciones:

\begin{figure}[H]
   \centering
    \begin{minipage}{0.45\textwidth}
        \centering
        \includegraphics[width=\textwidth]{imagenes/conv(s1).png}
        \caption{conv(s1)}
        \label{fig:grafico1}
    \end{minipage}
    \hfill
    \begin{minipage}{0.45\textwidth}
        \centering
        \includegraphics[width=\textwidth]{imagenes/conv(s2).png}
        \caption{conv(s2)}
        \label{fig:grafico2}
    \end{minipage}
\begin{minipage}{0.45\textwidth}
        \centering
        \includegraphics[width=\textwidth]{imagenes/conv(s3).png}
        \caption{conv(s3)}
        \label{fig:grafico3}
    \end{minipage}
    \label{fig:tres_graficos}
\end{figure}

Como se puede ver, las combinaciones convexas siempre están contenidas en el polígono que forman los puntos combinados (marcado en cada caso, salvo el de $S_1$, en negro).

\subsection*{Segundo ítem:}
Las curvas de Bézier son combinaciones convexas ya que, como demostramos en el ejercicio 2, la suma de los coeficientes que multiplican a los puntos siempre suman 1 y son $\geq 0$ para todo t entre 0 y 1. Por lo tanto, como en el ítem anterior, los puntos que resulten de ellas siempre estarán dentro del polígono delimitado por los puntos combinados, es decir, los puntos de control. Se muestran algunos ejemplos (generados corriendo varias veces el código del archivo .ipynb en la sección "Ejercicio 3"):

\begin{figure}[H]
   \centering
    \begin{minipage}{0.45\textwidth}
        \centering
        \includegraphics[width=\textwidth]{imagenes/3b1.png}
        \caption{Ejemplo 1 Bézier con 3 puntos}
        \label{fig:grafico1}
    \end{minipage}
    \hfill
    \begin{minipage}{0.45\textwidth}
        \centering
        \includegraphics[width=\textwidth]{imagenes/3b2.png}
        \caption{Ejemplo 2 Bézier con 3}
        \label{fig:grafico2}
    \end{minipage}
\begin{minipage}{0.45\textwidth}
        \centering
        \includegraphics[width=\textwidth]{imagenes/3b3.png}
        \caption{Ejemplo 1 Bézier con 4 puntos}
        \label{fig:grafico3}
    \end{minipage}
\begin{minipage}{0.45\textwidth}
        \centering
        \includegraphics[width=\textwidth]{imagenes/3b4.png}
        \caption{Ejemplo 2 Bézier con 4 puntos}
        \label{fig:grafico3}
    \end{minipage}
    \label{fig:tres_graficos}
\end{figure}

\textbf{Observación:} En caso de que el polígono formado por los puntos de control se cruce con la curva, es posible dibujar un nuevo polígono delimitado por los puntos de control que contenga a la curva en su totalidad.

\section*{Ejercicio 4:}
Para escribir la forma paramétrica de $x(t)$ como  $x(t) = G M_B u(t)$ donde $u(t) =
\left[ 
\begin{array}{c}
1 \\
t \\
t^2 \\
t^3
\end{array}
\right]
$, primero la desarrollamos obteniendo:

$$
x(t) = p_0 + t(-3p_0+3p_1) + t^2(3p_0-6p_1+3p_2) + t^3(-p_0+3p_1-3p_2+p_3)
$$

Con esta forma se puede ver más claramente como descomponer a  $x(t)$ como  $x(t) = G M_B u(t)$. La descomposición obtenida fue:

$$
x(t) = 
\begin{bmatrix}
p_0 & p_1 & p_2 & p_3
\end{bmatrix}
\begin{bmatrix}
1 & -3 & 3 & -1 \\
0 & 3 & -6 & 3 \\
0 & 0 & 3 & -3 \\
0 & 0 & 0 & 1
\end{bmatrix}
\begin{bmatrix}
1 \\
t \\
t^2 \\
t^3
\end{bmatrix}
$$

Para calcular $x'(t)$ y $x''(t)$  derivamos el vector $u(t)$ con respecto a $t$. De esta forma obtuvimos $u'(t) =
\begin{bmatrix}
0 \\ 1 \\ 2t \\ 3t^2
\end{bmatrix}
$ y $u''(t) =
\begin{bmatrix}
0 \\ 0 \\ 2 \\ 6t
\end{bmatrix}
$

Con esto, $x'(t) = G M_B u'(t)$ y $x''(t) = G M_B u''(t)$. Se desarrollaron estas expresiones como se muestra a continuación para llegar a una expresión desarrollada de $x'(t)$ y $x''(t)$.

\begin{figure}[H]
   \centering
    \begin{minipage}{0.45\textwidth}
        \centering
        \includegraphics[width=\textwidth]{imagenes/primderiv.jpg}
        \caption{Desarrollo de $x'(t)$}
        \label{fig:grafico1}
    \end{minipage}
    \hfill
    \begin{minipage}{0.45\textwidth}
        \centering
        \includegraphics[width=\textwidth]{imagenes/segderiv.jpg}
        \caption{Desarrollo de $x''(t)$}
        \label{fig:grafico2}
    \end{minipage}
    \label{fig:dos_graficos}
\end{figure}

Así, las expresiones desarrolladas para $x'(t)$ y $x''(t)$ son:

\begin{itemize}
    \item $x'(t) = p_0(-3 + 6t - 3t^2) + p_1(3 - 12t + 9t^2) + p_2(6t - 9t^2) + p_3(3t^2)$
    \item $x''(t) = p_0(6 - 6t) + p_1(-12 + 18t) + p_2(6 - 18t) + p_3(6t)$
\end{itemize}

Para determinar cómo se relaciona el vector tangente de la curva de Bézier x(t) con los puntos de control de la curva en $t=0$ y $t=1$, evaluamos $x'(t)$ en $t = 1$ y $t = 0$, obteniendo $x'(0) = 3p_1-3p_0$ y $x'(1) = 3p_2-3p_2$. Los vectores tangentes son:
\begin{itemize}
    \item En $t = 0$, $p_0 + (3p_1-3p_0) = 3p_1-2p_0$ (porque el vector pasa por el punto $p_0$).
    \item En $t = 1$, $p_3 +  (3p_3-3p_2) =  4p_3-3p_2$ (porque el vector pasa por el punto $p_3$).
\end{itemize}

A continuación mostramos un ejemplo de una curva de Bézier (la definida por los puntos $p_0 = (1, 6)$, $p_1 = (4, 9)$, $p_2 = (3, 9)$, $p_0 = (1, 0)$) y sus vectores tangentes en $t = 0$ y $t = 1$. El código que genera el gráfico est[a en el .ipynb en la sección "Ejercicio 4".

\begin{figure}[H]
    \centering
    \includegraphics[width=0.6\textwidth]{imagenes/ej4.png}
    \caption{Ejemplo de vectores tangentes a curva de Bézier}
    \label{fig:ejemplo}
\end{figure}

\section*{Ejercicio 5:}
\subsection*{Primer ítem}
Para que la curva compuesta por $x(t)$ e $y(t)$ sea $C^1$, necesitamos $x'(1) = y'(0)$, para garantizar la existencia y continuidad de la derivada de la curva compuesta en el punto donde se unen $x(t)$ e $y(t)$, que es $p_3 = x(1) = y(0)$. Del ejercicio 4, sabemos que $x'(1) = 3p_3-3p_2$ y que $y'(0) = 3p_4-3p_3$ (haciendo los reemplazos necesarios de los puntos de control, es decir $p_3$ como $p_0, p_4$ como $p_1$, etc).

Entonces, debe cumplirse,

$$
\begin{aligned}
x'(1) &= y'(0) \\
3p_3 - 3p_2 &= 3p_4 - 3p_3 \\
2p_3 &= p_4 +p_2
\end{aligned}
$$

\subsection*{Segundo ítem}
Cuando $x'(1) = y'(0) = 0$, se cumple que $x'(1) = 3p_3 - 3p_2 = 0$ por lo que $p_3 = p_2$. Similarmente, $y'(0) = 3p_4 - 3p_3 = 0$ por lo que $p_4 = p_3$. Entonces, cuando $x'(1) = y'(0) = 0$, se cumple $p_2 = p_3 = p_4$.

\subsection*{Tercer ítem}
Si la curva tiene continuidad $C^2$, debe cumplirse que tiene continuidad $C^1$ y que la derivada segunda de la curva compuesta es continua en $p_3$. Es decir, queremos $x''(1) = y''(0)$.

Evaluando las derivadas segundas de $x(t)$ e $y(t)$ en $t = 1$ y $t=0$ respectivamente, obtenemos:
\begin{itemize}
    \item $x''(1) = 6p_3 -12p_2 + 6p_1$
    \item $y''(0) = 6p_5 -12p_4 + 6p_3$
\end{itemize}

Entonces, para que la curva sea $C^2$ debe cumplirse:
$$
\begin{aligned}
x''(1) &= y''(0) \\
6p_3 -12p_2 + 6p_1 &= 6p_5 -12p_4 + 6p_3 \\
6p_5 &= 12p_4-12p_2+6p_1
\end{aligned}
$$

Pero como la curva también es $C^1$, sabemos que $2p_3 = p_4 +p_2$, por lo que $p_4 = 2p_3-p_2$. Entonces:

$$
\begin{aligned}
6p_5 &= 12p_4-12p_2+6p_1 \\
6p_5 &= 12(2p_3-p_2)-12p_2+6p_1 \\
p_5 &= 4p_3 - 4p_2 +p_1
\end{aligned}
$$

Así, vemos que los puntos $p_0$, ..., $p_5$ están determinados por $p_0$, $p_1$, $p_2$ y $p_3$, siendo  $p_4 = 2p_3-p_2$ y $p_5 = 4p_3 - 4p_2 +p_1$

\section*{Ejercicio 6}
\subsection*{Primer ítem}
La función implementada y el código que genera $x(t)$ e $y(t)$ se encuentran en la sección "Ejercicio 6" del archivo .ipynb. En la misma sección se generan las curvas de $x(t)$ e $y(t)$. Los gráficos resultantes de dichas curvas se muestran en el cuarto ítem de este ejercicio.

\subsection*{Segundo ítem}
Partimos de la igualdad $x(t) = (1 - t)^3 p_0 + 3t(1 - t)^2 p_1 + 3t^2(1 - t) p_2 + t^3 p_3$ y aplicamos la transformación $T(x)$ de ambos lados de la igualdad:

$$
\begin{aligned}
x(t) &= (1 - t)^3 p_0 + 3t(1 - t)^2 p_1 + 3t^2(1 - t) p_2 + t^3 p_3 \\
T(x(t)) &= T((1 - t)^3 p_0 + 3t(1 - t)^2 p_1 + 3t^2(1 - t) p_2 + t^3 p_3)
\end{aligned}
$$

Por linealidad de las transformaciones lineales, y teniendo en cuenta que los coeficientes polinomiales dependientes de $t$ son escalares para la transformación, vale que:

$$
\begin{aligned}
T(x(t)) &= T((1 - t)^3 p_0 + 3t(1 - t)^2 p_1 + 3t^2(1 - t) p_2 + t^3 p_3) \\
w(t) &= (1 - t)^3 T(p_0) + 3t(1 - t)^2 T(p_1) + 3t^2(1 - t) T(p_2) + t^3 T(p_3)
\end{aligned}
$$

Así, dada una transformación lineal $T(x)$, aplicarla sobre una curva de Bézier $x(t)$ resulta en en una nueva curva de Bézier $w(t)$ cuyos puntos de control serán  $T(p_0)$,  $T(p_1)$,  $T(p_2)$ y  $T(p_3)$.

\subsection*{Tercer ítem}
Definimos $T(x) = Ax = b$. De esta forma, buscamos que $T(x)$ cumpla, $T(p_0) = q_3$ y $T(p_3) = q_0$. 
Para eso, definimos: 

$A =  
\begin{bmatrix}
a & b \\
c & d 
\end{bmatrix}$ ,  
$p_0 =  
\begin{bmatrix}
p_{01} \\
p_{02} 
\end{bmatrix}$ , 
$p_3 =  
\begin{bmatrix}
p_{31} \\
p_{32} 
\end{bmatrix}$ , 
$q_0 =  
\begin{bmatrix}
q_{01} \\
q_{02} 
\end{bmatrix}$ , 
$q_3 =  
\begin{bmatrix}
q_{31} \\
q_{32} 
\end{bmatrix}$ \\

Luego, planteamos el sistema como se muestra en la siguiente imagen:

\begin{figure}[H]
    \centering
    \includegraphics[width=0.6\textwidth]{imagenes/6c.jpg}
    \caption{Planteo del sistema de ecuaciones}
    \label{fig:ejemplo}
\end{figure}

Así, llegamos al sistema 

$$
\begin{pmatrix}
p_{01} & p_{02} & 0 & 0 \\
0 & 0 & p_{01} & p_{02} \\
p_{31} & p_{32} & 0 & 0 \\
0 & 0 & p_{31} & p_{32}
\end{pmatrix}
\begin{pmatrix}
a \\
b \\
c \\
d
\end{pmatrix}
=
\begin{pmatrix}
q_{31} \\
q_{32} \\
q_{01} \\
q_{02}
\end{pmatrix}
$$\\

Resolviéndolo, llegamos a la solución

$a = \frac{-p_{32}q_{31} + p_{02}q_{01}}{p_{02}p_{31} - p_{01}p_{32}} $ , $b = \frac{p_{31}q_{31} - p_{01}q_{01}}{p_{02}p_{31} - p_{01}p_{32}} $ , $c = \frac{-p_{32}q_{32} + p_{02}q_{02}}{p_{02}p_{31} - p_{01}p_{32}} $ , $d = \frac{p_{31}q_{32} - p_{01}q_{02}}{p_{02}p_{31} - p_{01}p_{32}} $ \\

Como se puede ver, para que las soluciones existan, no puede pasar que $p_{02}p_{31} = p_{01}p_{32}$. Además, durante la resolución del sistema, $p_{01}$ es denominador en distintos momentos, por lo que como restricción adicional, no puede pasar  que $p_{01} = 0$.

Así, $T(x)$ existe solamente si $p_{02}p_{31} \neq p_{01}p_{32}$ y $p_{01} \neq 0$, y es:

$$ T(x) = 
\begin{pmatrix}
\frac{-p_{32}q_{31} + p_{02}q_{01}}{p_{02}p_{31} - p_{01}p_{32}} & \frac{p_{31}q_{31} - p_{01}q_{01}}{p_{02}p_{31} - p_{01}p_{32}}\\
\\
\frac{-p_{32}q_{32} + p_{02}q_{02}}{p_{02}p_{31} - p_{01}p_{32}} & \frac{p_{31}q_{32} - p_{01}q_{02}}{p_{02}p_{31} - p_{01}p_{32}}
\end{pmatrix}
x
$$

\subsection*{Cuarto ítem}
Definimos una función \textit{matriz\_transformacion(puntos\_p, puntos\_q)} en python que toma los puntos de control de $x(t)$ e $y(t)$ en \textit{puntos\_p} y \textit{puntos\_q} respectivamente y genera la matriz $A$ tal como fue definida en el ítem anterior. La misma se encuentra en la sección "Ejercicio 6" del archivo .ipynb. 

Además, en dicha sección se encuentra el resto del código necesario para generar y graficar $x(t)$, $y(t)$ y $T(x(t))$. El mismo consiste en:
\begin{itemize}
    \item Definir los puntos de control de $x(t)$ e $y(t)$.
    \item Generar y graficar las curvas $x(t)$ e $y(t)$.
    \item Calcular la matriz de transformación $A$.
    \item Generar $T(x(t))$ calculando el producto $A\begin{pmatrix}p_0, p_1, p_2, p_3 \end{pmatrix}$
    \item Graficar $T(x(t))$ junto con $y(t)$ y sus respectivos puntos de control.
\end{itemize}

Los gráficos resultantes son los siguientes: 

\begin{figure}[H]
   \centering
    \begin{minipage}{0.45\textwidth}
        \centering
        \includegraphics[width=\textwidth]{imagenes/6a1.png}
        \caption{Curva de Bézier aleatoria $x(t)$}
        \label{fig:grafico1}
    \end{minipage}
    \hfill
    \begin{minipage}{0.45\textwidth}
        \centering
        \includegraphics[width=\textwidth]{imagenes/6a2.png}
        \caption{Curva de Bézier aleatoria $y(t)$}
        \label{fig:grafico2}
    \end{minipage}
\begin{minipage}{0.45\textwidth}
        \centering
        \includegraphics[width=\textwidth]{imagenes/6d.png}
        \caption{Unión de $T(x(t))$ e $y(t)$}
        \label{fig:grafico2}
    \end{minipage}
    \label{fig:dos_graficos}
\end{figure}

\textbf {Observación:} en este caso es posible definir $T(x(t))$ ya que los puntos $\begin{pmatrix}p_0, p_1, p_2, p_3 \end{pmatrix}$ generados aleatoriamente cumplían las condiciones  $p_{02}p_{31} \neq p_{01}p_{32}$ y $p_{01} \neq 0$. En los casos en los que los puntos generados aleatoriamente no cumplan estas condiciones, la función \textit{matriz\_transformacion(puntos\_p, puntos\_q)} se encarga de devover \textit{None} y el programa termina con un mensaje comunicando que fue imposible calcular $T(x(t))$.

\end{document}


